\documentclass[a4paper, 12 pt]{article}
\usepackage[margin=1cm]{geometry}
\usepackage{mathrsfs}
\usepackage{braket}
\usepackage{amsmath}
\usepackage{mathrsfs}


\newcommand{\conc}[1]{\lbrack #1 \rbrack}
\newcommand{\deriv}[2]{\frac{d #1}{d #2}}
\newcommand{\pderiv}[2]{\frac{\partial #1}{\partial #2}}

\author{Alia Lescoulie}
\title{\vspace{-1.5cm}Kinetics}
\date{\today}

\begin{document}
\maketitle
\begin{flushleft}
\section*{Reaction Rates}

For a given reaction $R \rightarrow P$ the \textbf{instantaneous rate of consumption} and the  \textbf{instantaneous rate of formation} are both defined respectively in terms of derivatives bellow.

\begin{equation*}
    -\deriv{\conc{R}}{t} = \deriv{\conc{P}}{t} 
\end{equation*}

Therefor it follows that for a reaction $ A + 2B \rightarrow 3C + D$ has rates:

\begin{equation*}
    \deriv{\conc{D}}{t} = \frac{1}{3} \deriv{\conc{C}}{t} = -\deriv{\conc{A}}{t} = - \frac{1}{2} \deriv{\conc{B}}{t}
\end{equation*}

By introducing a parameter for \textbf{Extent of reaction} ($\xi$) the rate of reaction can be described with a single rate. For each species $J$ in reaction, the change in amount of $J$ $dn_J$ is given by:

\begin{equation*}
    dn_J = v_j d\xi
\end{equation*}

where $v_j$ is the stoichiometric number of the species, and negative for reactant species. From that the rate of reaction $\nu$ is defined as:

\begin{equation*}
    \nu = \frac{1}{V} \deriv{\xi}{t}
\end{equation*}

where $V$ is the volume of the system. For any species $J$ in the system, since $d\xi = dn_J/v_j$ its rate is given by:

\begin{equation*}
    \nu = \frac{1}{v_J} \times \frac{1}{V} \deriv{n_j}{t}
\end{equation*}

In the case of a homogenous constant volume reaction the volume is taken into the derivative giving $\deriv{\conc{J}}{t}$

\section*{Rate Laws}

Rates of reaction are often proportional to the concentrations of of the reactants raised to a power times a proportionality constant. An experimentally determined equation of this kind is the \textbf{Rate Law}.
Many reaction have rate laws of the form $\nu = k_r \conc{A}^a \conc{B}^b$ where $a$ and $b$ are the orders. The sum of the orders of the individual reactants is the \textbf{overall order}. Rate laws must be determined with experimental data. 

\end{flushleft}
\end{document}